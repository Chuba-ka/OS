\section{Результаты}

Разработана клиент-серверная система мгновенных сообщений, полностью соответствующая требованиям задания. Связь между сервером и клиентами реализована исключительно через механизм разделяемой памяти POSIX (\texttt{shm\_open} + \texttt{mmap}).

Ключевые особенности решения:

\begin{itemize}
  \item Сервер создаёт (при первом запуске) или открывает объект разделяемой памяти \texttt{/simple\_chat\_shm\_v1}, содержащий единую структуру \texttt{SharedData} с мьютексом и условной переменной (\texttt{PTHREAD\_PROCESS\_SHARED}), списком до 16 клиентов, кольцевым буфером на 1024 сообщения и служебными полями.

  \item Клиент подключается к существующей разделяемой памяти, вводит логин и регистрирует его в общем массиве клиентов.

  \item Отправка сообщения выполняется по логину получателя: клиент записывает сообщение в кольцевой буфер, обновляет \texttt{msg\_write\_index} и вызывает \texttt{pthread\_cond\_broadcast}.

  \item Приём сообщений в реальном времени обеспечивает отдельный поток в каждом клиенте: он ожидает на условной переменной, при пробуждении проверяет весь буфер, выводит новые сообщения для своего логина и помечает их как доставленные (\texttt{delivered = 1}).

  \item История переписки сохраняется на сервере в файле \texttt{history.txt}. Отдельный серверный поток при каждом срабатывании условной переменной записывает ещё не сохранённые сообщения в читаемом формате с меткой времени.

  \item Поиск по истории реализован через серверную команду \\
  \texttt{search <user1> <user2> [keyword]}. \\
  Команда выводит сообщения между указанными пользователями в обоих направлениях с возможностью фильтрации по ключевому слову.

  \item Корректное завершение работы по Ctrl+C и командам \texttt{quit}/\texttt{exit}: сервер выставляет флаг \texttt{server\_stopping}, делает \texttt{pthread\_cond\_broadcast}, клиенты и серверные потоки завершаются чисто.
\end{itemize}

Все пункты задания выполнены полностью. Система работает под Linux, обеспечивает мгновенную доставку сообщений и постоянное хранение полной истории переписки.