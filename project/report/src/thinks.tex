\section{Выводы}

В ходе курсовой работы полностью решена поставленная задача. Реализована простая клиент‑серверная система мгновенных сообщений с использованием POSIX shared memory (shm\_open + mmap) и межпроцессной синхронизации средствами pthread. Созданы отдельные процессы — сервер и клиент, обменивающиеся сообщениями через общую область памяти. Организована регистрация клиентов по логину и хранение таблицы клиентов в общей памяти. Реализована отправка сообщений по логину и приём сообщений в реальном времени на стороне клиента. Сервер запускает фоновый поток (pthreads), который сохраняет новые сообщения в файл истории \texttt{history.txt}. В серверном CLI реализованы команды просмотра списка клиентов и поиска по истории переписок. Обеспечен корректный механизм завершения работы: обработка сигнала \texttt{SIGINT}, установка флага \texttt{server\_stopping} в общей памяти и оповещение клиентов через \texttt{pthread\_cond\_broadcast}, вследствие чего клиенты корректно завершают работу. Проект собирается с помощью CMake и корректно работает под POSIX (проверено под Linux).

\pagebreak