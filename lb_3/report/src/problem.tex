\section{Условие}
        Родительский процесс создает два дочерних процесса. Первой строкой пользователь в консоль
родительского процесса вводит имя файла, которое будет использовано как отображаемый в память файл
для взаимодействия с child1. Аналогично для второй строки и процесса child2. Родительский 
и дочерний процессы представлены разными исполняемыми файлами. Родительский процесс принимает от пользователя строки произвольной длины и записывает их в
соответствующую область отображаемой памяти в зависимости от правила фильтрации. Процессы child1 и child2 читают данные из своих отображаемых файлов, обрабатывают строки и записывают результаты в файлы с суффиксом .out.

\subsection*{Цель работы}
        Изучение механизмов создания процессов, организации межпроцессного взаимодействия через отображаемые в память файлы (memory-mapped files) и обработки данных в многопроцессной архитектуре.

\subsection*{Задание}
    Правило фильтрации: строки длины больше 10 отправляются в отображаемый файл для child2, длиной 10 и менее — для child1.
    Дочерние процессы удаляют все гласные из строк.

\subsection*{Вариант}17