\section{Метод решения}
Данная программа реализует многопроцессную обработку текстовых данных с использованием отображаемых в память файлов (memory-mapped files) для межпроцессного взаимодействия.
Основной алгоритм: родительский процесс читает строки из стандартного ввода и записывает строки длиной больше 10 символов во второй отображаемый файл, длиной 10 и менее — в первый. Каждый дочерний процесс читает данные из своего отображаемого файла, удаляет все гласные из строк и записывает результат в файл с суффиксом .out.

Ключевые компоненты:\\
MappedFile — обёртка над системными вызовами mmap, munmap, open, ftruncate\\
ChildProcess — логика работы дочернего процесса\\
ChildProcessor — удаление гласных букв\\
\\
Системные вызовы:\\
Linux: mmap, munmap, open, ftruncate, fork, execl\\
\\
Программа использует объектно-ориентированный подход с инкапсуляцией.

\section{Описание программы}
Программа реализует многопроцессную обработку текстовых данных через отображаемые в память файлы. \\
Родительский процесс читает строки из стандартного ввода и распределяет их между двумя дочерними процессами по правилу длины. \\
Каждый дочерний процесс удаляет все гласные буквы и записывает результат в файл с суффиксом .out и в стандартный вывод.\\
\\
Программа состоит из двух исполняемых файлов: \texttt{parent} и \texttt{child}. \\
\texttt{parent} (\texttt{src/parent.cpp}) — родительский процесс: \\
создаёт отображаемые файлы (\texttt{MappedFile}), порождает дочерние процессы (\texttt{fork/execl}), \\
считывает строки и записывает их в соответствующую область памяти. \\
\texttt{child} (\texttt{src/child.cpp}) — точка входа дочернего процесса: \\
через \texttt{ChildProcess} опрашивает отображаемую память, \\
удаляет гласные с помощью \texttt{ChildProcessor} \\
и записывает результат в файл и на экран.