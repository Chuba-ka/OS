\section{Результаты}

Разработанная программа полностью соответствует поставленному варианту задания и успешно реализует расчёт приближённых значений чисел $\pi$ и $e$ двумя различными численными методами с использованием динамических библиотек.

В ходе выполнения лабораторной работы достигнуты следующие результаты:

Создано две динамические библиотеки, каждая из которых содержит две реализации:  
— первая библиотека: ряд Лейбница для $\pi$ и формула $(1 + 1/x)^x$ для $e$  
— вторая библиотека: формула (произведение) Валлиса для $\pi$ и сумма ряда Тейлора $\sum_{n=0}^{x} 1/n!$ для $e$  

Реализована программа №1 со статической линковкой на этапе компиляции (используется одна из библиотек)  

Реализована программа №2 с динамической загрузкой библиотек во время выполнения по относительным путям с помощью dlopen  

В программе №2 реализована смена активной библиотеки (а следовательно, обеих реализаций одновременно) по команде 0  

Организован интерактивный ввод в обеих программах строго в требуемом формате:  
— команда 1 и любое количество аргументов → вычисление $\pi$ с первым аргументом K  
— команда 2 и любое количество аргументов → вычисление $e$ с первым аргументом x  
— команда 0 → завершение в статической версии, переключение реализации в динамической  

Применена объектно-ориентированная парадигма: функции вычисления $\pi$ и $e$ вынесены в абстрактные базовые классы, конкретные алгоритмы реализованы через полиморфизм  

Программы успешно компилируются и работают под Linux с использованием CMake, демонстрируя корректную работу обоих способов подключения динамических библиотек.