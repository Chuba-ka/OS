\section{Условие}

Требуется реализовать вычисление приближённых значений чисел $\pi$ и $e$ двумя различными численными методами с использованием динамических библиотек и двух способов их подключения.

\subsection*{Цель работы}
Изучение механизмов создания и использования динамических библиотек в Linux, освоение статической и динамической линковки, применение объектно-ориентированного подхода при проектировании интерфейсов подключаемых модулей.

\subsection*{Задание}
Создать две динамические библиотеки, каждая из которых реализует два алгоритма:

\begin{itemize}
    \item Вычисление числа $\pi$ по заданному количеству членов ряда $K$:
    \begin{itemize}
        \item Ряд Лейбница: $\pi/4 = 1 - 1/3 + 1/5 - 1/7 + \dots$
        \item Формула (произведение) Валлиса
    \end{itemize}
    
    \item Вычисление числа $e$ по заданному параметру $x$:
    \begin{itemize}
        \item Формула предела: $e = \lim\limits_{x \to \infty} \left(1 + \frac{1}{x}\right)^x$
        \item Сумма ряда Тейлора: $e = \sum\limits_{n=0}^{x} \frac{1}{n!}$
    \end{itemize}
\end{itemize}

Реализовать две тестовые программы:
\begin{itemize}
    \item Программа №1 — использует одну из библиотек через статическую линковку на этапе компиляции
    \item Программа №2 — загружает библиотеки во время выполнения по относительным путям с помощью \texttt{dlopen}/\texttt{dlsym}, поддерживает переключение между реализациями по команде 0
\end{itemize}

Интерактивный ввод в обеих программах:
\begin{itemize}
    \item \texttt{1 arg1 arg2 $\dots$ argN} $\;\to\;$ вычисление $\pi$ с $K = \text{arg1}$
    \item \texttt{2 arg1 arg2 $\dots$ argM} $\;\to\;$ вычисление $e$ с $x = \text{arg1}$
    \item \texttt{0} $\;\to\;$ выход (в статической версии) или смена активной библиотеки (в динамической)
\end{itemize}

\subsection*{Вариант 25}