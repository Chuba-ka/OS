\section{Метод решения}

Программа решает задачу приближённого вычисления чисел $\pi$ и $e$ с использованием динамических библиотек и двух способов их подключения.

Применён объектно-ориентированный подход:
\begin{itemize}
    \item определены два абстрактных интерфейса — \texttt{PiCalculator} и \texttt{ECalculator};
    \item конкретные алгоритмы реализованы как классы-наследники этих интерфейсов.
\end{itemize}

Созданы две динамические библиотеки:
\begin{itemize}
    \item \texttt{lib1.so} — ряд Лейбница для $\pi$ и предел $(1 + 1/x)^x$ для $e$;
    \item \texttt{lib2.so} — произведение Валлиса для $\pi$ и ряд Тейлора $\sum_{n=0}^{x} 1/n!$ для $e$.
\end{itemize}

Для корректной динамической загрузки в каждой библиотеке реализованы четыре функции с привязкой \texttt{extern "C"}:
\texttt{create\_pi()}, \texttt{create\_e()}, \texttt{destroy\_pi()}, \texttt{destroy\_e()}.

Программа №1 использует одну из библиотек через статическую линковку на этапе компиляции.

Программа №2 загружает библиотеки во время выполнения с помощью \texttt{dlopen} по относительным путям \texttt{./lib1.so} и \texttt{./lib2.so}, получает указатели на функции через \texttt{dlsym} и управляет временем жизни объектов полиморфно. Переключение между парами реализаций происходит мгновенно по команде \texttt{0}.

Ввод реализован так: после команды \texttt{1} или \texttt{2} может следовать произвольное количество аргументов, используется только первый из них.

Сборка проекта выполняется с помощью системы CMake.

\section{Описание программы}

Проект состоит из двух исполняемых файлов и двух динамических библиотек:

\begin{itemize}
    \item \texttt{static\_app} — программа №1 \\
          статическая линковка с одной из библиотек, \\
          завершение работы по команде \texttt{0}

    \item \texttt{dynamic\_app} — программа №2 \\
          динамическая загрузка библиотек, \\
          переключение между реализациями по команде \texttt{0}, \\
          вывод названия текущего метода

    \item \texttt{lib1.so} и \texttt{lib2.so} — динамические библиотеки \\
          каждая содержит две реализации ($\pi$ и $e$), \\
          экспортируют единый интерфейс через функции \texttt{create\_*}/\texttt{destroy\_*}, \\
          компилируются с флагами \texttt{-shared -fPIC}
\end{itemize}
