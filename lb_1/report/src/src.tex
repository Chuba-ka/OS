\section{Метод решения}
Данная программа реализует многопроцессную обработку текстовых данных с использованием каналов (pipes) для межпроцессного взаимодействия.
Основной алгоритм: родительский процесс читает строки из стандартного ввода и направляет строки длиной больше 10 второму дочернему процессу, меньше 10 - первому. Каждый дочерний процесс получает строки из своего канала, удаляет все гласныеиз строк их и записывает в указанный файл.

Ключевые компоненты:\\
ParentProcess - управляет каналами и дочерними процессами\\
Pipe - реализация каналов\\
ChildProcess - запускает дочерние процессы\\
ChildProcessor - обрабатывает данные в дочерних процессах\\
\\
Системные вызовы:\\
Linux: pipe, fork, execl, read, write\\
\\
Программа использует объектно-ориентированный подход с инкапсуляцией.

\section{Описание программы}
Программа реализует многопроцессную обработку текстовых данных через каналы (pipes). \\
Родительский процесс читает строки из стандартного ввода и распределяет их между двумя дочерними процессами \\
Каждый дочерний процесс удаляет все гласные буквы и записывает результат в указанный файл.\\
\\
Программа состоит из двух исполняемых файлов: \texttt{parent} и \texttt{child}. \\
\texttt{parent} (\texttt{src/parent.cpp}) — родительский процесс: \\
создаёт каналы (\texttt{Pipe}), порождает дочерние процессы (\texttt{fork/execl}), \\
считывает строки до \texttt{END} и распределяет их по длине. \\
\texttt{child} (\texttt{src/child.cpp}) — дочерний процесс: \\
через \texttt{ChildProcess} читает данные из канала, \\
удаляет гласные с помощью \texttt{ChildProcessor} \\
и записывает результат в файл и на экран.\