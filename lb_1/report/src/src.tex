\section{Метод решения}
Данная программа реализует многопроцессную обработку текстовых данных с использованием каналов (pipes) для межпроцессного взаимодействия.
Основной алгоритм: родительский процесс читает строки из стандартного ввода и направляет строки длиной больше 10 второму дочернему процессу, меньше 10 - первому. Каждый дочерний процесс получает строки из своего канала, удаляет все гласныеиз строк их и записывает в указанный файл.

Ключевые компоненты:\\
ParentProcess - управляет каналами и дочерними процессами\\
Pipe - кросс-платформенная реализация каналов\\
ChildProcess - запускает дочерние процессы\\
ChildProcessor - обрабатывает данные в дочерних процессах\\
\\
Системные вызовы:\\
Linux: pipe, fork, execl, read, write\\
\\
Программа использует объектно-ориентированный подход с инкапсуляцией платформо-зависимых особенностей, что обеспечивает кроссплатформенность и четкое разделение ответственности между модулями.

\section{Описание программы}
Программа реализует многопроцессную обработку текстовых данных через каналы (pipes). \\
Родительский процесс читает строки из стандартного ввода и распределяет их между двумя дочерними процессами \\
Каждый дочерний процесс удаляет все гласные буквы и записывает результат в указанный файл.\\
\\
Архитектура программы включает несколько модулей. \\
В parent.cpp находится точка входа, создающая ParentProcess. \\
Класс ParentProcess (process.cpp) управляет всей работой: создает каналы, запрашивает имена файлов, запускает дочерние процессы и распределяет данные. \\
Класс Pipe (pipe.cpp) инкапсулирует работу с каналами, используя pipe на Linux. Класс ChildProcess (childProcess.cpp/hpp) отвечает за запуск дочерних процессов через CreateProcess (Windows) или fork/execl (Linux). \\
ChildProcessor (childProcessor.cpp/hpp) обрабатывает данные в дочерних процессах: читает из канала, переворачивает строки и записывает в файл.\\